\documentclass[12pt]{article}
\usepackage{design_ASC}
\usepackage[english]{babel}
\usepackage[utf8]{inputenc}
\usepackage[T1]{fontenc}
\usepackage{mathtools}
\usepackage{bm}

\setlength\parindent{0pt}

%% -----------------------------
%% -----------------------------
\title{Minimizaci\'on de AFD'S}

\author{Eduardo Salas\\ 
Jorge Nicho\\
Alvaro Aguirre\\
Teor\'ia de la computaci\'on\\ 
\textsc{UTEC}
}

\date{\today}
%% -----------------------------
%% -----------------------------

%% %%%%%%%%%%%%%%%%%%%%%%%%%
\begin{document}
\setlength{\droptitle}{-5em}    
%% %%%%%%%%%%%%%%%%%%%%%%%%%
\maketitle

% --------------------------
% --------------------------

% %%%%%%%%%%%%%%%%%%%
\section*{Introducci\'on}
% %%%%%%%%%%%%%%%%%%%
Podemos particionar los estados de cualquier AFD en grupos de mutuos indistinguibles estados. Los miembros de dos grupos diferentes son siempre distinguibles.  Si reemplazamos esos grupos por un \'unico estado, resulta en un AFD equivalente al inicial pero con una cantidad de estados m\'inimos.\\
\section*{Definici\'on del problema}
% %%%%%%%%%%%%%%%%%%%

La construcci\'on de un AFD no nos asegura el mejor rendimiento, ya que tiene dos o m\'as estados que podr\'ian ser un solo estado. Existen AFD’s que pueden ser minimizados y ofrecernos las siguientes ventajas:
\begin{enumerate}
    \item Ejecuci\'on: Debido a que se podr\'ia reducir la cantidad de estados de un AFD, este no demorar\'ia tanto en procesar una cadena, por lo cual disminuir\'ia su tiempo de ejecuci\'on.
    \item Costo: Cuanto mayor sea el n\'umero de estados mayor ser\'a el costo.
    \item Complejidad: Tener un menor n\'umero de estados sin afectar el lenguaje de un AFD nos permite tener un dise\~no m\'as limpio.
\end{enumerate}

\section{Estado del Arte}
%%%%%%%%%%%%%%%%%%%%%%

Dado un AFD M = (Q,$\sum$ ,$\delta$ , q0, F), se trata de encontrar un AFD M' con L(M) = L(M' )   y tal que M' tenga el m\'inimo n\'umero de estados posible.
El m\'etodo consiste en encontrar todos los estados que son equivalentes, es decir, que son indistinguibles en el aut\'omata. Por cada clase de estados equivalentes, el aut\'omata m\'inimo necesitar\'a un solo estado.
Hemos hallado 3 algoritmos para lograr este objetivo: Hopcroft, Moore y Myhill Nerode.

\subsection{Algoritmo de Hopcroft}

Hopcroft desarroll\'o en 1970 un algoritmo de minimizaci\'on que se ejecuta en tiempo
O(nlogn)en un aut\'omata de n estados. No se conoce otro algoritmo m\'as r\'apido que funcione para aut\'omatas en general.

A = (Q, i, F) aut\'omata en el alfabeto A. Sea P una partici\'on de Q.

Un divisor es un par (P, a), con P $\in$ P y a $\in$ A. El objetivo de un divisor es dividir otra clase de P.

El divisor (P, a) divide la clase R $\in$ P si $\varnothing$ ( P $\cap$ R · a ( R · a o equivalentemente si $\varnothing$ ( a −1P $\cap$ R ( R . Aqu\'i a−1P = {q | q · a $\in$ P}. 

En cualquier caso, denotamos por (P, a)|R la partici\'on de R compuesto por los conjuntos no vac\'ios de a −1P $\cap$ R and R $\setminus$ a −1P. El divisor (P, a) divide R si (P, a)|R 6= {R}.
\subsubsection{Ejemplo}
\begin{figure}[h]
    \centering
    \includegraphics{teo1.png}
    \caption{Ejemplo algoritmo de Hopcroft}
    \label{fig:my_label}
\end{figure}
P es la partición actual y W es el conjunto de espera.
\begin{enumerate}
    \item Partici\'on inicial P: 05|12346 
    \item Conjunto de espera W : (05, a), (05, b) 
    \item Eligiendo divisor : (05, a) 
    \item Estado divisor : a - 105 = 06 
    \item Primera clase en dividirse: 12346 $\rightarrow$ 1234|6 
    \item Divisores para agregar: (6, a) and (6, b) 
    \item Segunda clase en dividirse: 05 $\rightarrow$ 0|5 
    \item Divisores para agregar: (5, a) (or (0, a)) 
    \item Divisores para reemplazar: (05, b) : by (0, b) and (5, b) 
    \item Nueva partición P: 0|1234|5|6 
    \item Nuevo conjunto de espera W: (0, b), (6, a), (6, b), (5, a), (5, b)

\end{enumerate}

\subsubsection{Pseudocódigo de hopcroft}
\begin{figure}[h]
    \centering
    \includegraphics[width = 15 cm]{teo2.png}
    \caption{Pseudocódigo hopcroft}
    \label{fig:my_label}
\end{figure}

\newpage
\subsection{Algoritmo de Moore}
El m\'etodo para minimizar un aut\'omata mediante el algoritmo de moore consiste en encontrar todos los estados que son indistinguibles entre s\'i y sustituirlos por un \'unico estado. Para ello lo principal es averiguar qu\'e estados lo son y cu\'ales no. El m\'etodo para saber qu\'e estados son indistinguibles es el siguiente: \\
a. Si hay alg\'un estado inalcanzable eliminarlo.\\
b. (i := 0) Marcar todos los estados que pueden distinguirse con la cadena vac\'ia (es decir, todos los finales se pueden distinguir de los no finales).\\
c. (i := i + 1) Marcar como distinguibles q y q' si con alg\'un a $\in$ $\sum$ tenemos $\delta$(q, a) y $\delta$(q' , a) dos estados que ahora son distinguibles.\\ 
d. Si en el paso anterior se han distinguido nuevos estados, entonces volver al paso.\\

\subsubsection{Ejemplo}
\begin{figure}[h]
    \centering
    \includegraphics[width= 13 cm]{teo3.png}
    \caption{Aut\'omata ejemplo}
    \label{fig:my_label}
\end{figure}

Paso 1: Separamos los estados de aceptación y los de no aceptación.
\begin{figure}[h]
    \centering
    \includegraphics[width = 10 cm]{tep4.png}
    \caption{Paso 1}
    \label{fig:my_label}
\end{figure}

El conjunto de estados lo llamaremos Po siendo k = 0 ; quiere decir que tienen 0 transiciones iguales, solo están divididos por aceptación y no aceptación.
\newpage
Paso 2: Separamos aquellos estados en cada conjunto de Po que tenga 1 estado de transición distinguible.

\begin{figure}[h]
    \centering
    \includegraphics[width = 10 cm]{teo5.png}
    \caption{Paso 2}
    \label{fig:my_label}
\end{figure}

 Por lo tanto en el primer conjunto, todos los valores son indistinguibles. Quiere decir que este no será particionado en P1. 
 Ahora veamos el segundo conjunto:

\begin{figure}[h]
    \centering
    \includegraphics[width = 10 cm]{teo6.png}
    \caption{}
    \label{fig:my_label}
\end{figure}

Los movimientos de q0 con 0 y q3 con 0 son q3 y q0 que están en el mismo conjunto de partición en Po. Lo mismo sucede en q0 con 1 y q3 con 1, que son q1 y q4 , que también están en la misma partición en Po.
Caso contrario sucede con q5, veamos:

\begin{figure}[h]
    \centering
    \includegraphics[width = 10 cm]{teo7.png}
    \caption{}
    \label{fig:my_label}
\end{figure}
\\
Vemos que no pertenecen al mismo conjunto en Po. Es por ello que P1 quedaría de la siguiente manera:

\begin{figure}[h]
    \centering
    \includegraphics[width = 9 cm]{teo8.png}
\end{figure}    


\newpage
Este es el caso final puesto que no se pueden realizar más particiones. El autómata quedaría de la siguiente manera.

\begin{figure}[h]
    \centering
    \includegraphics[width = 15 cm]{teo9.png}
    \caption{A\'utomata final}
    \label{fig:my_label}
\end{figure}

\subsection{Algoritmo de Myhill Nerode}

Primero dibujamos una tabla para todos los pares de estados (P, Q). Luego marcamos los pares donde P sea un estado final y Q no lo sea o viceversa. Para los pares no marcados, analizamos sus estados de transici\'on. Por ejemplo [$\delta$(P,0), $\delta$(Q,0)]. Si est\'a marcado el nuevo par de estados a los que nos dirigimos con 0, entonces marcamos el par (P,Q). De lo contrario seguimos analizando los otros pares no marcados hasta que no se pueda marcar nada m\'as. Finalmente combinamos todos los pares no marcados y formamos un único estado en el AFD minimizado.

\begin{figure}[h]
    \centering
    \includegraphics[width = 13 cm]{teo10.png}
    \label{fig:my_label}
\end{figure}


\end{document}
